\documentclass[11pt,aspectratio=169]{beamer}% Normal
%\documentclass[11pt,aspectratio=169,handout]{beamer}% Für Handout
\usetheme{CambridgeUS}
% AnnArbor | Antibes | Bergen | Berkeley | Berlin | Boadilla | boxes | CambridgeUS | Copenhagen | Darmstadt |
% default | Dresden | Frankfurt | Goettingen | Hannover | Ilmenau | JuanLesPins | Luebeck | Madrid | Malmoe |
% Marburg |	Montpellier | PaloAlto | Pittsburgh | Rochester | Singapore | Szeged |	Warsaw

%\usecolortheme{beaver}	% Farbtheme
%\useinnertheme{beaver}	% inneres Theme (Titelseite, Umgebungen wie Aufzählungen und Blöcke,…)
%\useoutertheme{beaver}	% äußeres Theme (Kopf- und Fußzeile, Sidebars, Folientitel, Logo)
%\usefonttheme{beaver}	% Schrifttheme

\usepackage[utf8]{inputenc}
%\usepackage[english]{babel}
\usepackage{amsmath}
\usepackage{amsfonts}
\usepackage{rotating}
\usepackage{pgfgantt}
\usepackage{tabu}
\usepackage{tikz}
\usetikzlibrary{arrows}
\usetikzlibrary{positioning}
\usepackage{eurosym}
\usepackage{verbatim}
\usepackage{amssymb}
\usepackage{etoolbox}
\usepackage{lipsum}
\usepackage{graphicx}
\usepackage{lmodern}
\usepackage{amssymb}
\usepackage{wasysym}

\title{
\textcolor{cl1}{\Huge SUPER TOLLES THEMA}\newline
\textcolor{cl1!50}{\normalsize MIT ULTRA KRASSEN FEATURES}\newline
\textcolor{cl1!50}{\tiny IRGENDWO}
}

\author{\tiny Dockhorn}
\date{\vspace*{-0.5cm}\tiny\today{}}
\usepackage{template}







\begin{document}
%###########################################################################################################################################
\begin{frame}
%\vspace{-1cm}
\includegraphics[width=\textwidth]{logo/top.png}\\\vspace{1cm}
%\includegraphics[height=1cm]{logo/left.png}\hfill\includegraphics[height=1cm]{logo/right.png}\hspace{0.2cm}
\maketitle
\thispagestyle{empty}
\end{frame}
%###########################################################################################################################################
\section{SECTION 1}
\begin{frame}
\frametitle{TITEL 1}

\begin{itemize}			
	\item<1-> erster Punkt $\CheckedBox$
	\begin{itemize}			
	\item<2->  erster Punkt
		\begin{itemize}			
		\item<3->  erster Punkt
		\item<4->  zweiter Punkt
		\end{itemize}
		\item<5->  zweiter Punkt
	\end{itemize}
	\item<6->  zweiter Punkt
\end{itemize}
 
\begin{enumerate}		
	\item erster Punkt (1.)
	\item zweiter Punkt (2.)
	\begin{enumerate}
		\item Verschachtelung (2.1)
		\item zweiter Unterpunkt (2.2)
	\end{enumerate}
	\item dritter Punkt (3.)
\end{enumerate}

\end{frame}
%###########################################################################################################################################
\begin{frame}
\frametitle{TITEL 2}
\begin{block}{BLOCK 1}

\end{block}
\begin{alertblock}{BLOCK 2}

\end{alertblock}
\end{frame}
%###########################################################################################################################################
\section{SECTION 2}
\begin{frame}
\frametitle{TITEL 3}

\end{frame}
%###########################################################################################################################################
\section{}
\begin{frame}
\begin{center}
{\Huge { \textcolor{cl1}{Thank you!}}}
\end{center}
\end{frame}
%###########################################################################################################################################
\end{document}
